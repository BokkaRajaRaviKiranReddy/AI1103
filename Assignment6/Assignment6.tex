\documentclass[journal,12pt,twocolumn]{IEEEtran}

\usepackage{setspace}
\usepackage{gensymb}
\singlespacing
\usepackage[cmex10]{amsmath}
\usepackage{amssymb}
\usepackage{xurl}
\usepackage{tabularx}
\usepackage{amsthm}
\usepackage{comment}
\usepackage{mathrsfs}
\usepackage{txfonts}
\usepackage{stfloats}
\usepackage{bm}
\usepackage{cite}
\usepackage{cases}
\usepackage{subfig}

\usepackage{longtable}
\usepackage{multirow}

\usepackage{enumitem}
\usepackage{mathtools}
\usepackage{steinmetz}
\usepackage{tikz}
\usepackage{circuitikz}
\usepackage{verbatim}
\usepackage{tfrupee}
\usepackage[breaklinks=true]{hyperref}
\usepackage{graphicx}
\usepackage{tkz-euclide}

\usetikzlibrary{calc,math}
\usepackage{listings}
    \usepackage{color}                                            %%
    \usepackage{array}                                            %%
    \usepackage{longtable}                                        %%
    \usepackage{calc}                                             %%
    \usepackage{multirow}                                         %%
    \usepackage{hhline}                                           %%
    \usepackage{ifthen}                                           %%
    \usepackage{lscape}     
\usepackage{multicol}
\usepackage{chngcntr}

\DeclareMathOperator*{\Res}{Res}

\renewcommand\thesection{\arabic{section}}
\renewcommand\thesubsection{\thesection.\arabic{subsection}}
\renewcommand\thesubsubsection{\thesubsection.\arabic{subsubsection}}

\renewcommand\thesectiondis{\arabic{section}}
\renewcommand\thesubsectiondis{\thesectiondis.\arabic{subsection}}
\renewcommand\thesubsubsectiondis{\thesubsectiondis.\arabic{subsubsection}}


\hyphenation{op-tical net-works semi-conduc-tor}
\def\inputGnumericTable{}                                 %%

\lstset{
%language=C,
frame=single, 
breaklines=true,
columns=fullflexible
}
\begin{document}


\newtheorem{theorem}{Theorem}[section]
\newtheorem{problem}{Problem}
\newtheorem{proposition}{Proposition}[section]
\newtheorem{lemma}{Lemma}[section]
\newtheorem{corollary}[theorem]{Corollary}
\newtheorem{example}{Example}[section]
\newtheorem{definition}[problem]{Definition}

\newcommand{\BEQA}{\begin{eqnarray}}
\newcommand{\EEQA}{\end{eqnarray}}
\newcommand{\define}{\stackrel{\triangle}{=}}
\bibliographystyle{IEEEtran}
\raggedbottom
\setlength{\parindent}{0pt}
\providecommand{\mbf}{\mathbf}
\providecommand{\pr}[1]{\ensuremath{\Pr\left(#1\right)}}
\providecommand{\qfunc}[1]{\ensuremath{Q\left(#1\right)}}
\providecommand{\sbrak}[1]{\ensuremath{{}\left[#1\right]}}
\providecommand{\lsbrak}[1]{\ensuremath{{}\left[#1\right.}}
\providecommand{\rsbrak}[1]{\ensuremath{{}\left.#1\right]}}
\providecommand{\brak}[1]{\ensuremath{\left(#1\right)}}
\providecommand{\lbrak}[1]{\ensuremath{\left(#1\right.}}
\providecommand{\rbrak}[1]{\ensuremath{\left.#1\right)}}
\providecommand{\cbrak}[1]{\ensuremath{\left\{#1\right\}}}
\providecommand{\lcbrak}[1]{\ensuremath{\left\{#1\right.}}
\providecommand{\rcbrak}[1]{\ensuremath{\left.#1\right\}}}
\theoremstyle{remark}
\newtheorem{rem}{Remark}
\newcommand{\sgn}{\mathop{\mathrm{sgn}}}
\providecommand{\abs}[1]{\vert#1\vert}
\providecommand{\res}[1]{\Res\displaylimits_{#1}} 
\providecommand{\norm}[1]{\lVert#1\rVert}
%\providecommand{\norm}[1]{\lVert#1\rVert}
\providecommand{\mtx}[1]{\mathbf{#1}}
\providecommand{\mean}[1]{E[ #1 ]}
\providecommand{\fourier}{\overset{\mathcal{F}}{ \rightleftharpoons}}
%\providecommand{\hilbert}{\overset{\mathcal{H}}{ \rightleftharpoons}}
\providecommand{\system}{\overset{\mathcal{H}}{ \longleftrightarrow}}
	%\newcommand{\solution}[2]{\textbf{Solution:}{#1}}
\newcommand{\solution}{\noindent \textbf{Solution: }}
\newcommand{\cosec}{\,\text{cosec}\,}
\providecommand{\dec}[2]{\ensuremath{\overset{#1}{\underset{#2}{\gtrless}}}}
\newcommand{\myvec}[1]{\ensuremath{\begin{pmatrix}#1\end{pmatrix}}}
\newcommand{\mydet}[1]{\ensuremath{\begin{vmatrix}#1\end{vmatrix}}}
\newcommand*{\permcomb}[4][0mu]{{{}^{#3}\mkern#1#2_{#4}}}
\newcommand*{\perm}[1][-3mu]{\permcomb[#1]{P}}
\newcommand*{\comb}[1][-1mu]{\permcomb[#1]{C}}
\numberwithin{equation}{subsection}
\makeatletter
\@addtoreset{figure}{problem}
\makeatother
\let\StandardTheFigure\thefigure
\let\vec\mathbf
\renewcommand{\thefigure}{\theproblem}
\def\putbox#1#2#3{\makebox[0in][l]{\makebox[#1][l]{}\raisebox{\baselineskip}[0in][0in]{\raisebox{#2}[0in][0in]{#3}}}}
     \def\rightbox#1{\makebox[0in][r]{#1}}
     \def\centbox#1{\makebox[0in]{#1}}
     \def\topbox#1{\raisebox{-\baselineskip}[0in][0in]{#1}}
     \def\midbox#1{\raisebox{-0.5\baselineskip}[0in][0in]{#1}}
\vspace{3cm}
\title{AI1103 : Assignment 6}
\author{Raja Ravi Kiran Reddy - CS20BTECH11009}
\maketitle
\newpage
\bigskip
\renewcommand{\thefigure}{\arabic{figure}}
\renewcommand{\thetable}{\arabic{table}}
%
Download Latex codes from 
%
\begin{lstlisting}
https://github.com/BokkaRajaRaviKiranReddy/AI1103/blob/main/Assignment6/Assignment6.tex
\end{lstlisting}
\section{UGC NET 2013 JUNE Q.NO 101}
Let $X_1$,$X_2$,...be the independent random variables each following exponential distribution with mean 1.then which of the following statements are correct?
\begin{enumerate}
    \item \pr{X_n>\log n \quad \forall n\geq1}=1\\
    \item \pr{X_n>2\quad \forall n\geq1}=1\\
    \item \pr{X_n>\frac{1}{2} \quad \forall n\geq1}=0\\
    \item  \pr{X_n>\log n,X_{n+1}>\log (n+1)\quad\forall n\geq1}=0 
\end{enumerate}
\section{SOLUTION}
\begin{lemma}\label{pdf and mean}
PDF of $X_i$ is
\begin{align}
    f_{X_i}(x)=\begin{cases}\lambda_i e^{-\lambda_i x}, &x\geq 0\\
                0, &x<0\nonumber
    \end{cases}    
\end{align}
Mean of $X_i=\frac{1}{\lambda_i}$\\

\end{lemma}
\begin{proof}
\begin{align}
    \mean{X_i}&=\int_{-\infty}^{\infty}x f_{X_i}(x) dx\\
              &=\int_{-\infty}^{0}0 dx + \int_{0}^{\infty}x \lambda_i e^{-\lambda_i x}\\
              &=\frac{1}{\lambda_i}\label{mean}
\end{align}
\end{proof}
\begin{lemma}\label{cdf}
  CDF of  $X_i$ is
\begin{align}
    F_{X_i}(x)=\begin{cases}1-e^{-\lambda_i x}, &x\geq 0\\
                0, &x<0\nonumber
    \end{cases}    
\end{align}
\end{lemma}
\begin{proof}
\begin{align}
   F_{X_i}(x)=\int_{-\infty}^{x}f_{X_i}(x)dx
\end{align}
For $x<0$
\begin{align}
    F_{X_i}(x)&=\int_{-\infty}^{x}0dx\\
    &=0
\end{align}
For $x\geq0$
\begin{align}
    F_{X_i}(x)&=\int_{-\infty}^{x}f_{X_i}(x)dx\\
    &=\int_{-\infty}^{0}0dx +\int_{0}^{x}\lambda_i e^{-\lambda_i x}dx\\
    &=1-e^{-\lambda_i x}
\end{align}
so,
\begin{align}
    F_{X_i}(x)=\begin{cases}1-e^{-\lambda_i x}, &x\geq 0\\
                0, &x<0\nonumber
    \end{cases}    
\end{align}
\end{proof}



From \eqref{mean} and $\mean{X_i}=1$, we have $\lambda_i=1 \forall  i \geq1$
Now, for some constant $c\geq0$ 
\begin{align}
    \pr{X_n>c}&=1-F_{X_n}(c)\\
             &=e^{-c}\label{probability}
\end{align}
We need Borel-Cantelli Lemmas
\begin{lemma}\label{Lemma1}
Let $A_1$,$A_2$,... be a sequence of events in some probability space. The Borel–Cantelli lemma states that, if the sum of the probabilities of the events $A_n$ is finite
\begin{align}
    \sum_{n=1}^{\infty}\pr{A_n}&<\infty
\end{align}
then the probability that infinitely many of them occur is 0
\begin{align}
    \Pr{\lim_{n \rightarrow \infty}\sup A_n}&=0
\end{align}
\end{lemma}
\begin{lemma}\label{Lemma2}
If the events $A_n$ are independent and the sum of the probabilities of the $A_n$ diverges to infinity, then the probability that infinitely many of them occur is 1.
If for independent events $A_1,A_2,...$
\begin{align}
    \sum_{n=1}^{\infty}\pr{A_n}&=\infty
\end{align}
Then
\begin{align}
    \pr{\lim_{n \rightarrow \infty}\sup A_n}&=1
\end{align}
\end{lemma}
\begin{enumerate}
    \item OPTION 1: 
    We can say the events $X_n>\log n$ are independent $\forall n\geq 1$ as $X_n$ are independent random variable.
    
   from \eqref{probability}
    \begin{align}
        \sum_{n=1}^{\infty}\pr{X_n > \log n} &=\sum_{n=1}^{\infty}e^{-\log n}\\ &=\sum_{n=1}^{\infty}\frac{1}{n}\\
        &= \infty\text{ (Cauchy's Criterion)}  
    \end{align}
    Now, from \ref{Lemma2}
    \begin{align}
        &\pr{X_n>\log n\hspace{0.2cm} \forall n\geq1}\\
        &=\pr{\lim_{n \rightarrow \infty}\sup X_n>\log n}\\
        &=1
    \end{align}
    $\therefore$ Option 1 is correct. 
    
    \item OPTION 2: We can say the events $X_n>2$ are independent $\forall n\geq 1$ as $X_n$ are independent random variable.
    
    From \eqref{probability}
    \begin{align}
        \sum_{n=1}^{\infty}\pr{X_n > 2} &= \sum_{n=1}^{\infty}e^{-2}\\
                                            &= \infty
    \end{align}
    Now, from \ref{Lemma2}
    \begin{align}
        &\pr{X_n>2 \hspace{0.2cm} \forall n\geq1}\\
        &=\pr{\lim_{n \rightarrow \infty}\sup X_n>2}\\
        &=1
    \end{align}
    $\therefore$ Option 2 is correct.
    
    \item OPTION 3: We can say the events $X_n>\frac{1}{2}$ are independent $\forall n\geq 1$ as $X_n$ are independent random variable.
    
    From \eqref{probability}
    \begin{align}
        \sum_{n=1}^{\infty}\pr{X_n > \frac{1}{2}} &= \sum_{n=1}^{\infty}e^{-\frac{1}{2}}\\
                                            &= \infty
    \end{align}
    Now, from \ref{Lemma2}
    \begin{align}
        &\pr{X_n>\frac{1}{2} \hspace{0.2cm} \forall n\geq1}\\
        &=\pr{\lim_{n \rightarrow \infty}\sup X_n>\frac{1}{2}}\\
        &=1
    \end{align}
    $\therefore$ Option 3 is incorrect.
    \item OPTION 4:We can say the events $X_n>\log n$ are independent $\forall n\geq 1$ as $X_n$ are independent random variable.
    
    Let the event $X_n > \log n,X_{n+1}>\log (n+1)$ be represented by $A_n$'
    
    From \eqref{probability}
    \begin{align}
        &\sum_{n=1}^{\infty}\pr{A_n}\\
        &= \sum_{n=1}^{\infty}\pr{X_n>\log n}\pr{X_{n+1}>\log (n+1)}\\
        &=\sum_{n=1}^{\infty}e^{-\log n}e^{-\log (n+1)}\\
        &=\sum_{n=1}^{\infty}\frac{1}{n(n+1)}\\
        &=\sum_{n=1}^{\infty}\frac{1}{n}-\frac{1}{n+1}\\
        &=1
    \end{align}
    Now, from \ref{Lemma1}
    \begin{align}
        &\pr{X_n>\log n,X_{n+1}>\log (n+1)\hspace{0.2cm}\forall n\geq1}\\
        &=\pr{\lim_{n \rightarrow \infty}\sup ( X_n>\log n,X_{n+1}>\log (n+1))}\\
        &=0
    \end{align}
    $\therefore$ Option 4 is correct.\\
    Correct options are \textbf{Option 1,2,4}
\end{enumerate}
\end{document}
