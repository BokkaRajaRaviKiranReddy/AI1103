\documentclass[journal,12pt,twocolumn]{IEEEtran}

\usepackage{setspace}
\usepackage{gensymb}
\singlespacing
\usepackage[cmex10]{amsmath}
\usepackage{amssymb}
\usepackage{xurl}
\usepackage{tabularx}
\usepackage{amsthm}
\usepackage{comment}
\usepackage{mathrsfs}
\usepackage{txfonts}
\usepackage{stfloats}
\usepackage{bm}
\usepackage{cite}
\usepackage{cases}
\usepackage{subfig}

\usepackage{longtable}
\usepackage{multirow}

\usepackage{enumitem}
\usepackage{mathtools}
\usepackage{steinmetz}
\usepackage{tikz}
\usepackage{circuitikz}
\usepackage{verbatim}
\usepackage{tfrupee}
\usepackage[breaklinks=true]{hyperref}
\usepackage{graphicx}
\usepackage{tkz-euclide}

\usetikzlibrary{calc,math}
\usepackage{listings}
    \usepackage{color}                                            %%
    \usepackage{array}                                            %%
    \usepackage{longtable}                                        %%
    \usepackage{calc}                                             %%
    \usepackage{multirow}                                         %%
    \usepackage{hhline}                                           %%
    \usepackage{ifthen}                                           %%
    \usepackage{lscape}     
\usepackage{multicol}
\usepackage{chngcntr}

\DeclareMathOperator*{\Res}{Res}

\renewcommand\thesection{\arabic{section}}
\renewcommand\thesubsection{\thesection.\arabic{subsection}}
\renewcommand\thesubsubsection{\thesubsection.\arabic{subsubsection}}

\renewcommand\thesectiondis{\arabic{section}}
\renewcommand\thesubsectiondis{\thesectiondis.\arabic{subsection}}
\renewcommand\thesubsubsectiondis{\thesubsectiondis.\arabic{subsubsection}}


\hyphenation{op-tical net-works semi-conduc-tor}
\def\inputGnumericTable{}                                 %%

\lstset{
%language=C,
frame=single, 
breaklines=true,
columns=fullflexible
}
\begin{document}


\newtheorem{theorem}{Theorem}[section]
\newtheorem{problem}{Problem}
\newtheorem{proposition}{Proposition}[section]
\newtheorem{lemma}{Lemma}[section]
\newtheorem{corollary}[theorem]{Corollary}
\newtheorem{example}{Example}[section]
\newtheorem{definition}[problem]{Definition}

\newcommand{\BEQA}{\begin{eqnarray}}
\newcommand{\EEQA}{\end{eqnarray}}
\newcommand{\define}{\stackrel{\triangle}{=}}
\bibliographystyle{IEEEtran}
\raggedbottom
\setlength{\parindent}{0pt}
\providecommand{\mbf}{\mathbf}
\providecommand{\pr}[1]{\ensuremath{\Pr\left(#1\right)}}
\providecommand{\qfunc}[1]{\ensuremath{Q\left(#1\right)}}
\providecommand{\sbrak}[1]{\ensuremath{{}\left[#1\right]}}
\providecommand{\lsbrak}[1]{\ensuremath{{}\left[#1\right.}}
\providecommand{\rsbrak}[1]{\ensuremath{{}\left.#1\right]}}
\providecommand{\brak}[1]{\ensuremath{\left(#1\right)}}
\providecommand{\lbrak}[1]{\ensuremath{\left(#1\right.}}
\providecommand{\rbrak}[1]{\ensuremath{\left.#1\right)}}
\providecommand{\cbrak}[1]{\ensuremath{\left\{#1\right\}}}
\providecommand{\lcbrak}[1]{\ensuremath{\left\{#1\right.}}
\providecommand{\rcbrak}[1]{\ensuremath{\left.#1\right\}}}
\theoremstyle{remark}
\newtheorem{rem}{Remark}
\newcommand{\sgn}{\mathop{\mathrm{sgn}}}
\providecommand{\abs}[1]{\vert#1\vert}
\providecommand{\res}[1]{\Res\displaylimits_{#1}} 
\providecommand{\norm}[1]{\lVert#1\rVert}
%\providecommand{\norm}[1]{\lVert#1\rVert}
\providecommand{\mtx}[1]{\mathbf{#1}}
\providecommand{\mean}[1]{E[ #1 ]}
\providecommand{\fourier}{\overset{\mathcal{F}}{ \rightleftharpoons}}
%\providecommand{\hilbert}{\overset{\mathcal{H}}{ \rightleftharpoons}}
\providecommand{\system}{\overset{\mathcal{H}}{ \longleftrightarrow}}
	%\newcommand{\solution}[2]{\textbf{Solution:}{#1}}
\newcommand{\solution}{\noindent \textbf{Solution: }}
\newcommand{\cosec}{\,\text{cosec}\,}
\providecommand{\dec}[2]{\ensuremath{\overset{#1}{\underset{#2}{\gtrless}}}}
\newcommand{\myvec}[1]{\ensuremath{\begin{pmatrix}#1\end{pmatrix}}}
\newcommand{\mydet}[1]{\ensuremath{\begin{vmatrix}#1\end{vmatrix}}}
\newcommand*{\permcomb}[4][0mu]{{{}^{#3}\mkern#1#2_{#4}}}
\newcommand*{\perm}[1][-3mu]{\permcomb[#1]{P}}
\newcommand*{\comb}[1][-1mu]{\permcomb[#1]{C}}
\numberwithin{equation}{subsection}
\makeatletter
\@addtoreset{figure}{problem}
\makeatother
\let\StandardTheFigure\thefigure
\let\vec\mathbf
\renewcommand{\thefigure}{\theproblem}
\def\putbox#1#2#3{\makebox[0in][l]{\makebox[#1][l]{}\raisebox{\baselineskip}[0in][0in]{\raisebox{#2}[0in][0in]{#3}}}}
     \def\rightbox#1{\makebox[0in][r]{#1}}
     \def\centbox#1{\makebox[0in]{#1}}
     \def\topbox#1{\raisebox{-\baselineskip}[0in][0in]{#1}}
     \def\midbox#1{\raisebox{-0.5\baselineskip}[0in][0in]{#1}}
\vspace{3cm}
\title{AI1103 : Assignment 3}
\author{Raja Ravi Kiran Reddy - CS20BTECH11009}
\maketitle
\newpage
\bigskip
\renewcommand{\thefigure}{\arabic{figure}}
\renewcommand{\thetable}{\arabic{table}}

Download the latex codes from 

\begin{lstlisting}
https://github.com/BokkaRajaRaviKiranReddy/AI1103/blob/main/Assignment3/Assignment3.tex
\end{lstlisting}
\section*{GATE 2014 MA -Q37}
Let $X ,Y$ be continuous random variables with joint density function
\begin{align*}
    f_{X,Y}(x,y)=\begin{cases}
    e^{-y}(1-e^{-x}) \text{   if } 0< x<y<\infty\\
    e^{-x}(1-e^{-y}) \text{   if } 0< y\leq x<\infty
    \end{cases}
\end{align*}
Then The value of $E[X+Y]$ is 
\section*{SOLUTION}
Let $g(X,Y)=X+Y$

We know that,
\begin{align*}
    &E[g(X,Y)]=\int_{-\infty}^{+\infty}\int_{-\infty}^{+\infty}g(x,y)f_{X,Y}(x,y)dxdy\\
\end{align*}
Then,
\begin{align*}
    &E[X+Y]=\int_{-\infty}^{+\infty}\int_{-\infty}^{+\infty}(x+y)f_{X,Y}(x,y)\,dxdy\\
    &=\int_{0}^{+\infty}\int_{0}^{+\infty}(x+y)f_{X,Y}(x,y)\,dxdy\\
    &=\int_{0}^{+\infty}\left(\int_{0}^{+\infty}xf_{X,Y}(x,y)\,dx+\int_{0}^{+\infty}yf_{X,Y}(x,y)\,dx\right)\,dy
\end{align*}
First we will calculate the $\int_{0}^{+\infty}yf_{X,Y}(x,y)\,dx$,  $\int_{0}^{+\infty}xf_{X,Y}(x,y)\,dx$ seperately.\\
consider,
\begin{align*}
    &\int_{0}^{+\infty}yf_{X,Y}(x,y)\,dx\\
    &=\int_{0}^{y}ye^{-y}(1-e^{-x})\,dx+\int_{y}^{+\infty}ye^{-x}(1-e^{-y})\,dx\\
    &=(ye^{-y})(y+e^{-y}-1)+y(1-e^{-y})e^{-y}\\
    &=y^2e^{-y}
\end{align*}
So,
\begin{align}
    \tag{37.1}
    \int_{0}^{+\infty}yf_{X,Y}(x,y)\,dx=y^2e^{-y}
\end{align}
Now consider,
\begin{align*}
  &\int_{0}^{+\infty}xf_{X,Y}(x,y)\,dx\\
  &=\int_{0}^{y}xe^{-y}(1-e^{-x})\,dx+\int_{y}^{+\infty}xe^{-x}(1-e^{-y})\\
  &=e^{-y}\left(\frac{y^2}{2}+e^{-y}(y+1)-1\right)+(1-e^{-y})(e^{-y}(y+1))\\
  &=\frac{y^2e^{-y}}{2}+ye^{-y}
\end{align*}
So,
\begin{align}
    \tag{37.2}
    \int_{0}^{+\infty}xf_{X,Y}(x,y)\,dx=\frac{y^2e^{-y}}{2}+ye^{-y}
\end{align}
From Eq 37.1 and 37.2
\begin{align*}
    &E[X+Y]=\int_{0}^{+\infty}\left(\frac{y^2e^{-y}}{2}+ye^{-y}+y^2e^{-y}\right)\,dy\\
    &=\int_{0}^{+\infty}\left(\frac{3}{2}y^2e^{-y}+ye^{-y}\right)\,dy\\
    &=\left(-\frac{3}{2}(y^2+2y+2)e^{-y}+(-e^{-y}(y+1))\right)\Biggr|_{0}^{+\infty}\\
    &=\frac{3}{2}\times2+1\\
    &=4
\end{align*}
So,
\begin{align*}
  &E[X+Y]=4
\end{align*}
\end{document}